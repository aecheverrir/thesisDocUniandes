% !TEX root = ../thesis-example.tex
%
\chapter{Introduction}
\label{sec:intro}

\section{Introduction}

During the development of a system, performing tests is needed in order to validate that it behaves as intended. Therefore, through testing, it is possible to validate many of the functional and nonfunctional characteristics of your system. However, hundreds of testing types and testing techniques exist for this purpose.

Regression testing is a common type of testing that often faces problems. This kind of testing involves performing a series of previously executed tests in order to validate that a change in the system has not affected any of its dependencies. In situations where the system is relatively large or it experiences frequent changes, performing regression testing can become a complication. This is due to the fact that the bigger the project or the higher the frequency of changes in it, more testing will be required, and more testing means increasing costs and delays in adapting the desired changes \cite{Google Automate UI tests}. Here is where the concept of test automation becomes quite beneficial, especially in continuous integration projects \cite{ISTQB}.

It is very common to see the previously mentioned situation during mobile application development. This is due to the fast-moving pace and the increasing size and complexity of projects in the current mobile application market. Not only do users expect frequent improvements, but they expect high quality apps. This can considerably influence the number of installs, user rating and as user retention. Therefore, it is crucial to assess the quality of your app with an efficient testing process \cite{Google App Quality}.

A proper and efficient testing process on a mobile application development project will most likely require the automation of several tests \cite{Gaurav Why Test Automation}. But, before commencing with the test scripting, a series of important decisions regarding the design of the test automation architecture must be made. These commonly involve choosing a testing framework and any other tools to be used. A common issue here is that this decision might not be as straight forward as one would desire, since it actually depends on a series of factors related to the project and the tool characteristics. Taking the wrong decision at this level may result extremely expensive and time consuming since it could unnecessarily increase the difficulty level, the time needed at the test automation implementation or even worse, it could require an architectural change mid-project, putting a lot of work in vain. 

For the purposes of this project, an approach will be made into the user interface test automation tools available for the iOS platform. It is specially attractive to conduct a research on this area due to the fact that this platform is way more limited compared to the Android platform in terms of available tools and operating system restrictions.

\section{Objectives}
Studying and analyzing a series of existing tools for the development of automated interface tests in mobile applications that run over the iOS operating system. This would allow identifying weaknesses, strengths, characteristics and limitations of existing tools in order to take the most convenient choice when designing the architecture of a test automation project.

\subsection{Specific Objective: Tool Research}
\begin{itemize}
	\item Conducting a research on the currently available tools that allow the automation of user interface tests in iOS applications. This means, defining what tools will be taken into account in the development of this project and which of their attributes are more relevant to compare and analyze.
\end{itemize}
\subsection{Specific Objective: Case Study}
\begin{itemize}
	\item Designing a case study that will allow the evaluation of the attributes defined in the previous phase of research, as well as defining any other attribute that is considered relevant during the development of this phase.
\end{itemize}
\subsection{Specific Objective: Documentation}
\begin{itemize}
	\item Publishing a report with the results of the case study in a document that describes the differences between the different tools. As a result, this document would explain their weaknesses, strengths, characteristics and limitations.
\end{itemize}

\section{Expected Results}
Upon completion of this project, a detailed documentation on a series of tools currently available for user interface test automation in iOS applications will be available. This documentation would have a comparison of the most relevant characteristics for each of the selected tools. On the other hand, the code developed for the case study will be published as open source in order to collaborate with the community. These two should ease the decision making when designing the architecture of an automation project.